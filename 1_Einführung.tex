\section{Einführung}     \marginpar{VL 1}

\subsection*{Worum geht es?}

Wie beschriebt man in der Physik ein System mit sehr vielen Teilchen? \\
\hspace*{20mm} $\Rightarrow$ Orte $\Vec{r_i}$, Impulse $\Vec{p_i}$ für alle $i = 1, \dots , N$ zu allen Zeiten $t$ \\
\hspace*{20mm} $\Rightarrow$ \textbf{NEIN!} $\rightarrow$ zu viel Detailinformation\\

\underline{relevante Größen - Ideensammlung}
\begin{itemize}
    \item Druck
    \item Temperatur, mittlere Energie
    \item Zahl der Teilchen / Teilchendichte($\Vec{r}$)
    \item Wirkungsquerschnitt der Teilchen
    \item mittlere freie Weglänge
    \item Verteilung der Impulse
\end{itemize}

\textbf{Prozesse in der Natur sind nicht reversibel!} Ein umgefallenes Glas Wasser kann nicht einfach mit einem Fingerschnipsen wieder in den Ausgangszustand zurückgeführt werden.

\subsection*{Ziel}
Beschreibung makroskopischer Eigenschaften von Systemen mit sehr vielen Freiheitsgraden aus mikroskopischen Gesetzen der Physik. Relevante Größen sollen hierbei Energie, Entropie, Volumen, Teilchenzahl, Druck, Temperatur etc. sein.

\subsection{Was ist Thermodynamik?}
\begin{itemize}
    \item makroskopische, phänomenologische Theorie aus Experimenten und Erfahrung (19. Jhd)
    \item Beschreibung von Vielteilchensystemen
    \item keine mikroskopischen Details $\Rightarrow$ allgemeingültig!
    \item Hauptsätze
    \item wichtige Begriffe: Temperatur, Entropie
\end{itemize}


\subsection{Was ist statistische Physik?}

\begin{itemize}
    \item mikroskopische Herleitung der Hauptsätze (unter gewissen Annahmen)
    \item Materialeigenschaften lassen sich  mikroskopisch herleiten
    \item klassische Mechanik und Quantenmechanik
    \item Entropie: Mangel an Informationen
    \item wechselwirkende und nichtwechselwirkende Teilchen
    \item Phasenübergänge
\end{itemize}

